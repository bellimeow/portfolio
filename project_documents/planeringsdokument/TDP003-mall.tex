\documentclass{TDP003mall}



\newcommand{\version}{Version 1.0}
\author{Isabella Delgado, \url{isade842@student.liu.se}\\
  Pia Løtvedt, \url{pialo059@student.liu.se}}
\title{Planeringsdokument}
\date{2017-09-04}
\rhead{Isabella Delgado\\
Pia Løtvedt}



\begin{document}
\projectpage

\section{Planeringsdokumentets syfte}
Planeringsdokumentet ska visa att vi har förstått vad projektet innebär, och att vi har tänkt igenom arbetet som kommer att ingå och hur lång tid olika uppgifter kan ta. Dokumentet kan användas som underlag till projektplanens tidsplanering.

\section{Projektets mål och innehåll}
Målet för projektet är att skapa en webbportfolio som kan visa upp våra olika projekt från IP-utbildningen. I projektet ska vi skapa hela systemet, från att hantera datat, dvs skapa ett datalager, till att presentera datat på en hemsida, dvs skapa ett presentationslager.

I presentationslagret ska det, kort berättat, finnas:

\begin{itemize}
\item Framsida
\item Lista över projekt
\item Lista över tekniker
\item Sidor för varje projekt
\end{itemize}


\medskip
I datalagret ska det finnas:
\begin{itemize}
\item En textfil som innehåller datat i JSON-format
\item En API som kan hantera datat och skicka vidare information.
\end{itemize}

\medskip

Förutom datalagret och presentationslagret ska vi även göra:

\begin{itemize}
\item Planeringsdokument (detta dokument)
\item LoFi-prototyp
\item Projektplan
\item En gemensam installationsmanual
\item Gemensamma tester
\item Systemdokumentation
\item Reflektionsdokument
\end{itemize}


\section{Generell tidplan}
Nedan följer en genomgång över kursens generella tidplan och deadlines.

\subsection*{Vecka 36}

\textbf{Fas:} Förberedelse\\

Vi ska börja planera projektet. Detta innebär att läsa noggrant igenom all information på kurshemsidan och få en uppfattning om vad projektet innebär.\\

\textbf{Deadlines:} Inlämning av planeringsdokumentet - torsdag 7 september.

\subsection*{Vecka 37}

\textbf{Fas:} Förberedelse\\

LoFi-prototypen ska vara en modell av vårt systems design, med fokus på användargränssnittet på hemsidan. Vi kommer behöva tänka igenom hur hemsidan bör se ut, och skapa ett presentation av detta. Prototypen kan vi göra med penna och papper, i powerpoint, eller som en enklare html-sida.
Installationsmanualen är, som namnet säger, en beskrivning av hur man kan installera systemet. Den här veckan ska vi bara lämna in en grundläggande version. Innan vi börjar med detta behöver vi ta reda på vad som egentligen ska vara med i den här versionen.
Vi bör också jobba med projektplanen den här veckan.\\


\textbf{Deadlines:} LoFi-prototyp, grundläggande installationsmanual - 14 september

\subsection*{Vecka 38}

\textbf{Fas:} Förberedelse, börja konstruktion\\

Den här veckan ska vi lämna in ett första utkast till den gemensamma installationsmanualen, samt ett första utkast till projektplanen.
Projektplanen är ett mycket viktigt dokument som ska innehålla i stort sett all information som är relevant för projektet. Dvs, vad som ska göras, vem som ska göra det, när det ska göras, vilka verktyg som ska användas osv. Projektplanen kommer att innehålla mycket detalj, och vi behöver sätta av tillräckligt med tid. 
Vi bör även jobba med vårt datalager den här veckan, då det ska vara klart nästa vecka.
Båda inlämningar ska ske senast torsdag 21 september.\\


\textbf{Deadlines:} Första utkast till gemensamma installationsmanualen, första utkast till projektplanen - 21 september

\subsection*{Vecka 39}

\textbf{Fas:} Konstruktion\\

Den här veckan ska vi vara klar med vårt datalager. Vi kommer därför fokusera på att få det färdigt den här veckan. Detta innebär att vi kommer jobba med att implementera dom 6 funktioner som ska vara datalagrets API. Parallellt bör vi även jobba med systemdokumentationen.
Vi kan även lämna in korrigerad projektplan och installationsmanual den här veckan. 
Datalagret ska godkännas innan kl 12 fredagen 29 september, medan korrigerade dokument ska in senast torsdagen 28 september.\\


\textbf{Deadlines:} Korrigerad projektplan, korrigerad installationsmanual - 28 september

\subsection*{Vecka 40}

\textbf{Fas:} Konstruktion\\

Vi har inget schemalagt den här veckan, vilket ger oss bra möjlighet att jobba koncentrerat med presentationslagret, som ska in veckan efter.\\

\textbf{Deadlines:} Inga


\subsection*{Vecka 41}

\textbf{Fas:} Konstruktion, överlämning\\

Innan den här veckan bör alla större delar av systemet vara klara. Den här veckan kan vi därmed fokusera på ingående testing, buggfixar, och mindre förbättringar av systemet. Vi kan även färdigställa systemdokumentationen.
Portfolion ska ligga på OpenShift senast torsdag den här veckan. Systemdokumentationen ska också in, och vi kommer att ha redovisning av vårt system.\\


\textbf{Deadlines:} Portfolion på OpenShift, inlämning av sytemdokumentation - 12 oktober


\subsection*{Vecka 42}

\textbf{Fas:} Uppföljning\\

Den här veckan ska de sista dokumenten in, dvs testdokumentation, reflektionsdokument, samt korrigerad systemdokumentation.\\


\textbf{Deadlines:} Reflektionsdokument, testdokumentation, korrigerad systemdokumentation - 19 oktober




\end{document}


